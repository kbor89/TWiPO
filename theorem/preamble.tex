\usetheme{Madrid}
\useoutertheme{miniframes} 
\useinnertheme{circles}

\hypersetup{pdfpagemode=FullScreen}
%\setbeameroption{show notes}
%\setbeameroption{show notes on second screen=right}

\usepackage[T1]{fontenc}
\usepackage[utf8]{inputenc}
\usepackage[polish]{babel}


\author[Krzysztof Borkowski]{dr inż. Krzysztof Borkowski}
\institute[tu.kielce.pl]{Politechnika Świętokrzyska \\  Wydział Mechatroniki i Budowy Maszyn \\ Katedra Automatyki i Robotyki} 
\date[2025/2026]{Rok akademicki 2025/2026} 


\usepackage{lmodern}
\usepackage{amsmath}
\usepackage{amsfonts}

\DeclareMathAlphabet{\mathpzc}{OT1}{pzc}{m}{it}
\DeclareMathAlphabet{\mathbfit}{OML}{cmm}{b}{it}

\DeclareMathAlphabet{\mathsca}{OML}{cmm}{m}{it}
\DeclareMathAlphabet{\mathvec}{OML}{cmm}{b}{it}
\DeclareMathAlphabet{\mathvech}{OT1}{cmr}{b}{it}
\DeclareMathAlphabet{\mathmatrix}{OT1}{cmss}{bx}{it}
\DeclareMathAlphabet{\mathname}{OT1}{pzc}{m}{it}

\usepackage{icomma}
\usepackage{amssymb}

\usefonttheme{professionalfonts}

\usepackage{listings} 
\lstset{
  language=C++,
%  basicstyle=\ttfamily\small,  
  basicstyle=\footnotesize\ttfamily,
  keywordstyle=\color{blue}\bfseries,
  commentstyle=\color{green!40!black}\itshape,
  stringstyle=\color{red},
  showstringspaces=false,
  literate={ą}{{\k{a}}}1
           {Ą}{{\k{A}}}1
           {ć}{{\'c}}1
           {Ć}{{\'C}}1
           {ę}{{\k{e}}}1
           {Ę}{{\k{E}}}1
           {ł}{{\l{}}}1
           {Ł}{{\L{}}}1
           {ń}{{\'n}}1
           {Ń}{{\'N}}1
           {ó}{{\'o}}1
           {Ó}{{\'O}}1
           {ś}{{\'s}}1
           {Ś}{{\'S}}1
           {ż}{{\.z}}1
           {Ż}{{\.Z}}1
           {ź}{{\'z}}1
           {Ź}{{\'Z}}1
}

\usepackage{tikz}  
\usetikzlibrary{arrows,shapes,positioning,fit,backgrounds}

\definecolor{myColorDepartment}{HTML}{163f8b}
\definecolor{myColorDepartment2}{HTML}{11406e}

\usecolortheme[named=myColorDepartment]{structure}

\graphicspath{{../}}

%\logo{\includegraphics[width=0.15\textwidth]{img/psk_logo.png}} 
%\newcommand{\nologo}{\setbeamertemplate{logo}{}}

	\AtBeginSubsection[]
{
	\begin{frame}
        \frametitle{\secname}
        \tableofcontents[sectionstyle=hide/hide,subsectionstyle=show/shaded/hide]
    \end{frame}
}

\newcommand\imgwidth{0.65\textwidth}
\newcommand{\figurewidth}{0.48}

\setbeamertemplate{caption}[numbered]
\setbeamerfont{caption}{size=\scriptsize}
\setbeamerfont{caption name}{size=\scriptsize}

%\setbeamerfont{bibliography entry author}{size=\tiny}
%\setbeamerfont{bibliography entry title}{size=\tiny}
%\setbeamerfont{bibliography entry journal}{size=\tiny}
%\setbeamerfont{bibliography entry note}{size=\tiny}

\def\figurename{Rys.}

\usepackage[skip=3pt]{subcaption} 

\usepackage{ifthen}
\newboolean{ifpdfimg}  
\setboolean{ifpdfimg}{false}

\usepackage{multicol}
\usepackage{makecell}

\newcommand{\rvect}[1]{\begin{bmatrix} #1 \end{bmatrix}^T}
\newcommand{\cvect}[1]{\begin{bmatrix} #1 \end{bmatrix}}
\newcommand{\cvector}[1]{\left[\begin{array}{c} #1 \end{array}\right]}


\newcommand*{\unit}[2][\;]{\ensuremath{#1\mathrm{#2}}}
\newcommand*{\unitb}[2][\;]{\ensuremath{#1\mathrm{[#2]}}}

\newcommand*{\li}[1]{\ensuremath{{\,}^{\mathrm{#1}\!}}}

\newcommand*{\coord}[2][]{\ensuremath{{#2}^{\mathrm{#1}}{}}}
\newcommand*{\vect}[2][]{\ensuremath{\mathvec{#2}^{\mathrm{#1}}{}}}
\newcommand*{\vecth}[2][]{\ensuremath{\mathvech{#2}^{\mathrm{#1}}{}}}

\newcommand*{\axis}[2][]{\ensuremath{ \mathmatrix{#2}_{\mathrm{#1}}}}
\newcommand*{\framexy}[3][]{\ensuremath{ \axis[#1]{#2} \axis[#1]{#3} }}
\newcommand*{\frameXYZ}[4][]{\ensuremath{ \axis[#1]{#2} \axis[#1]{#3} \axis[#1]{#4}}}


\newcommand{\source}[1]{\vspace{-6pt} \caption*{\hfill Źródło: {#1}}}

\usepackage{textcomp, gensymb}

\usepackage{multirow}
\usepackage{booktabs}

\newcommand{\head}[1]{%
  \bfseries
  \begin{tabular}{@{}c@{}}
  \strut#1\strut
  \end{tabular}%
}

%%	***	TODO LIST  ***
%\usepackage{enumitem,amssymb}
%\newlist{todolist}{itemize}{2}
%\setlist[todolist]{label=$\square$}
\usepackage{pifont}
\newcommand{\cmark}{\ding{51}}%
\newcommand{\xmark}{\ding{55}}%
\newcommand{\checked}{\rlap{$\square$}{\raisebox{2pt}{\large\hspace{1pt}\cmark}}\hspace{-2.5pt}}%
\newcommand{\wontfix}{\rlap{$\square$}{\large\hspace{1pt}\xmark}}


\usepackage{etoolbox}

\makeatletter
\patchcmd{\beamer@sectionintoc}{\vskip1.5em}{\vskip1.0em}{}{}
\makeatother

% Ustawienia dla listingów kodu
\lstset{
    language=C++,
    basicstyle=\footnotesize\ttfamily,
    keywordstyle=\color{blue}\bfseries,
    commentstyle=\color{green!50!black}\itshape,
    stringstyle=\color{red},
    numbers=left,
    numberstyle=\tiny\color{gray},
    stepnumber=1,
    numbersep=5pt,
    backgroundcolor=\color{gray!10},
    frame=single,
    frameround=tttt,
    captionpos=b,
    breaklines=true,
    breakatwhitespace=true,
    showstringspaces=false,
    tabsize=4,
    escapeinside={(*@}{@*)}
}


%\Huge
%\huge
%\LARGE
%\Large
%\large
%\normalsize (domyślny)
%\small
%\footnotesize
%\scriptsize
%\tiny

% w preambule
\setbeamerfont{block title}{size=\small}
\setbeamerfont{block body}{size=\small}

\setbeamerfont{alertblock title}{size=\small}
\setbeamerfont{alertblock body}{size=\small}

\setbeamerfont{exampleblock title}{size=\small}
\setbeamerfont{exampleblock body}{size=\small}