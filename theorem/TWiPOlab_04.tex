\documentclass[9pt]{beamer}

\usetheme{Madrid}
\useoutertheme{miniframes} 
\useinnertheme{circles}

\hypersetup{pdfpagemode=FullScreen}
%\setbeameroption{show notes}
%\setbeameroption{show notes on second screen=right}

\usepackage[T1]{fontenc}
\usepackage[utf8]{inputenc}
\usepackage[polish]{babel}


\author[Krzysztof Borkowski]{dr inż. Krzysztof Borkowski}
\institute[tu.kielce.pl]{Politechnika Świętokrzyska \\  Wydział Mechatroniki i Budowy Maszyn \\ Katedra Automatyki i Robotyki} 
\date[2025/2026]{Rok akademicki 2025/2026} 


\usepackage{lmodern}
\usepackage{amsmath}
\usepackage{amsfonts}

\DeclareMathAlphabet{\mathpzc}{OT1}{pzc}{m}{it}
\DeclareMathAlphabet{\mathbfit}{OML}{cmm}{b}{it}

\DeclareMathAlphabet{\mathsca}{OML}{cmm}{m}{it}
\DeclareMathAlphabet{\mathvec}{OML}{cmm}{b}{it}
\DeclareMathAlphabet{\mathvech}{OT1}{cmr}{b}{it}
\DeclareMathAlphabet{\mathmatrix}{OT1}{cmss}{bx}{it}
\DeclareMathAlphabet{\mathname}{OT1}{pzc}{m}{it}

\usepackage{icomma}
\usepackage{amssymb}

\usefonttheme{professionalfonts}

\usepackage{listings} 
\lstset{
  language=C++,
%  basicstyle=\ttfamily\small,  
  basicstyle=\footnotesize\ttfamily,
  keywordstyle=\color{blue}\bfseries,
  commentstyle=\color{green!40!black}\itshape,
  stringstyle=\color{red},
  showstringspaces=false,
  literate={ą}{{\k{a}}}1
           {Ą}{{\k{A}}}1
           {ć}{{\'c}}1
           {Ć}{{\'C}}1
           {ę}{{\k{e}}}1
           {Ę}{{\k{E}}}1
           {ł}{{\l{}}}1
           {Ł}{{\L{}}}1
           {ń}{{\'n}}1
           {Ń}{{\'N}}1
           {ó}{{\'o}}1
           {Ó}{{\'O}}1
           {ś}{{\'s}}1
           {Ś}{{\'S}}1
           {ż}{{\.z}}1
           {Ż}{{\.Z}}1
           {ź}{{\'z}}1
           {Ź}{{\'Z}}1
}

\usepackage{tikz}  
\usetikzlibrary{arrows,shapes,positioning,fit,backgrounds}

\definecolor{myColorDepartment}{HTML}{163f8b}
\definecolor{myColorDepartment2}{HTML}{11406e}

\usecolortheme[named=myColorDepartment]{structure}

\graphicspath{{../}}

%\logo{\includegraphics[width=0.15\textwidth]{img/psk_logo.png}} 
%\newcommand{\nologo}{\setbeamertemplate{logo}{}}

	\AtBeginSubsection[]
{
	\begin{frame}
        \frametitle{\secname}
        \tableofcontents[sectionstyle=hide/hide,subsectionstyle=show/shaded/hide]
    \end{frame}
}

\newcommand\imgwidth{0.65\textwidth}
\newcommand{\figurewidth}{0.48}

\setbeamertemplate{caption}[numbered]
\setbeamerfont{caption}{size=\scriptsize}
\setbeamerfont{caption name}{size=\scriptsize}

%\setbeamerfont{bibliography entry author}{size=\tiny}
%\setbeamerfont{bibliography entry title}{size=\tiny}
%\setbeamerfont{bibliography entry journal}{size=\tiny}
%\setbeamerfont{bibliography entry note}{size=\tiny}

\def\figurename{Rys.}

\usepackage[skip=3pt]{subcaption} 

\usepackage{ifthen}
\newboolean{ifpdfimg}  
\setboolean{ifpdfimg}{false}

\usepackage{multicol}
\usepackage{makecell}

\newcommand{\rvect}[1]{\begin{bmatrix} #1 \end{bmatrix}^T}
\newcommand{\cvect}[1]{\begin{bmatrix} #1 \end{bmatrix}}
\newcommand{\cvector}[1]{\left[\begin{array}{c} #1 \end{array}\right]}


\newcommand*{\unit}[2][\;]{\ensuremath{#1\mathrm{#2}}}
\newcommand*{\unitb}[2][\;]{\ensuremath{#1\mathrm{[#2]}}}

\newcommand*{\li}[1]{\ensuremath{{\,}^{\mathrm{#1}\!}}}

\newcommand*{\coord}[2][]{\ensuremath{{#2}^{\mathrm{#1}}{}}}
\newcommand*{\vect}[2][]{\ensuremath{\mathvec{#2}^{\mathrm{#1}}{}}}
\newcommand*{\vecth}[2][]{\ensuremath{\mathvech{#2}^{\mathrm{#1}}{}}}

\newcommand*{\axis}[2][]{\ensuremath{ \mathmatrix{#2}_{\mathrm{#1}}}}
\newcommand*{\framexy}[3][]{\ensuremath{ \axis[#1]{#2} \axis[#1]{#3} }}
\newcommand*{\frameXYZ}[4][]{\ensuremath{ \axis[#1]{#2} \axis[#1]{#3} \axis[#1]{#4}}}


\newcommand{\source}[1]{\vspace{-6pt} \caption*{\hfill Źródło: {#1}}}

\usepackage{textcomp, gensymb}

\usepackage{multirow}
\usepackage{booktabs}

\newcommand{\head}[1]{%
  \bfseries
  \begin{tabular}{@{}c@{}}
  \strut#1\strut
  \end{tabular}%
}

%%	***	TODO LIST  ***
%\usepackage{enumitem,amssymb}
%\newlist{todolist}{itemize}{2}
%\setlist[todolist]{label=$\square$}
\usepackage{pifont}
\newcommand{\cmark}{\ding{51}}%
\newcommand{\xmark}{\ding{55}}%
\newcommand{\checked}{\rlap{$\square$}{\raisebox{2pt}{\large\hspace{1pt}\cmark}}\hspace{-2.5pt}}%
\newcommand{\wontfix}{\rlap{$\square$}{\large\hspace{1pt}\xmark}}


\usepackage{etoolbox}

\makeatletter
\patchcmd{\beamer@sectionintoc}{\vskip1.5em}{\vskip1.0em}{}{}
\makeatother

% Ustawienia dla listingów kodu
\lstset{
    language=C++,
    basicstyle=\footnotesize\ttfamily,
    keywordstyle=\color{blue}\bfseries,
    commentstyle=\color{green!50!black}\itshape,
    stringstyle=\color{red},
    numbers=left,
    numberstyle=\tiny\color{gray},
    stepnumber=1,
    numbersep=5pt,
    backgroundcolor=\color{gray!10},
    frame=single,
    frameround=tttt,
    captionpos=b,
    breaklines=true,
    breakatwhitespace=true,
    showstringspaces=false,
    tabsize=4,
    escapeinside={(*@}{@*)}
}


%\Huge
%\huge
%\LARGE
%\Large
%\large
%\normalsize (domyślny)
%\small
%\footnotesize
%\scriptsize
%\tiny

% w preambule
\setbeamerfont{block title}{size=\small}
\setbeamerfont{block body}{size=\small}

\setbeamerfont{alertblock title}{size=\small}
\setbeamerfont{alertblock body}{size=\small}

\setbeamerfont{exampleblock title}{size=\small}
\setbeamerfont{exampleblock body}{size=\small}

\title[Transformata Census]{Techniki Wizyjne i Przetwarzanie Obrazów}
\subtitle{Laboratorium 4: Transformata Census}

\begin{document}
	
\begin{frame}[plain]
\titlepage
\note{
Witaj na czwartym laboratorium! Dziś poznasz transformatę Census - metodę reprezentacji lokalnych regionów obrazu w postaci ciągów binarnych.
}
\end{frame}

\begin{frame}[allowframebreaks]{Plan laboratorium}
\tableofcontents[hideallsubsections]
\note{
Transformata Census to potężne narzędzie w przetwarzaniu obrazów, szczególnie przydatne w stereo vision i detekcji zmian.
}
\end{frame}

\section{Wprowadzenie teoretyczne}

\subsection{Czym jest transformata Census?}

\begin{frame}{Transformata Census - idea}
\begin{block}{Definicja}
\textbf{Transformata Census} to metoda reprezentacji lokalnych regionów obrazu w postaci ciągów binarnych poprzez porównanie wartości piksela centralnego z jego sąsiadami.
\end{block}

\pause

\begin{block}{Zastosowania}
\begin{itemize}
\item \textbf{Stereo vision} - dopasowywanie punktów między obrazami
\item \textbf{Detekcja zmian} - wykrywanie różnic w sekwencjach obrazów
\item \textbf{Rozpoznawanie tekstur} - analiza struktury powierzchni
\item \textbf{Tracking obiektów} - śledzenie obiektów w ruchu
\end{itemize}
\end{block}

\pause

\begin{alertblock}{Kluczowa zaleta}
Transformata Census jest \textbf{odporna na zmiany oświetlenia}, ponieważ porównuje względne relacje między pikselami, a nie ich bezwzględne wartości.
\end{alertblock}

\note{
Census to elegancka metoda, która zamienia lokalny region obrazu na ciąg bitów. Jest szybka, efektywna i odporna na zmiany oświetlenia.
}
\end{frame}

\begin{frame}{Zasada działania}
\begin{center}
\begin{tikzpicture}[scale=0.9]
% Okno 3x3
\node[draw, rectangle, minimum width=4cm, minimum height=4cm, fill=blue!10] (window) at (0,0) {};
\node[above=2mm of window] {\textbf{Okno 3×3}};

% Piksel centralny
\node[draw, circle, fill=red!30, minimum size=1cm] (center) at (0,0) {50};
\node[below=-1mm of center] {\footnotesize centrum};

% Sąsiedzi
\node[draw, circle, fill=green!20, minimum size=0.8cm] at (-1.2,1.2) {45};
\node[draw, circle, fill=green!20, minimum size=0.8cm] at (0,1.2) {60};
\node[draw, circle, fill=green!20, minimum size=0.8cm] at (1.2,1.2) {30};
\node[draw, circle, fill=green!20, minimum size=0.8cm] at (-1.2,0) {55};
\node[draw, circle, fill=green!20, minimum size=0.8cm] at (1.2,0) {40};
\node[draw, circle, fill=green!20, minimum size=0.8cm] at (-1.2,-1.2) {70};
\node[draw, circle, fill=green!20, minimum size=0.8cm] at (0,-1.2) {35};
\node[draw, circle, fill=green!20, minimum size=0.8cm] at (1.2,-1.2) {65};

% Strzałka
\draw[->, thick] (3,0) -- (5,0);

% Wynik
\node[draw, rectangle, minimum width=4cm, minimum height=4cm, fill=yellow!10] (result) at (8,0) {};
\node[above=2mm of result] {\textbf{Porównanie}};

\node[draw, circle, fill=red!30, minimum size=1cm] at (8,0) {X};
\node[draw, circle, fill=blue!20, minimum size=0.8cm] at (6.8,1.2) {0};
\node[draw, circle, fill=orange!30, minimum size=0.8cm] at (8,1.2) {1};
\node[draw, circle, fill=blue!20, minimum size=0.8cm] at (9.2,1.2) {0};
\node[draw, circle, fill=orange!30, minimum size=0.8cm] at (6.8,0) {1};
\node[draw, circle, fill=blue!20, minimum size=0.8cm] at (9.2,0) {0};
\node[draw, circle, fill=orange!30, minimum size=0.8cm] at (6.8,-1.2) {1};
\node[draw, circle, fill=blue!20, minimum size=0.8cm] at (8,-1.2) {0};
\node[draw, circle, fill=orange!30, minimum size=0.8cm] at (9.2,-1.2) {1};
\end{tikzpicture}
\end{center}

\vspace{5mm}

\begin{columns}
\begin{column}{0.5\textwidth}
\textbf{Reguła:}
\begin{itemize}
\item Piksel $<$ centrum $\rightarrow$ bit = \textcolor{blue}{\textbf{0}}
\item Piksel $\geq$ centrum $\rightarrow$ bit = \textcolor{orange}{\textbf{1}}
\end{itemize}
\end{column}
\begin{column}{0.5\textwidth}
\textbf{Wynik:}
\begin{itemize}
\item Ciąg bitów: 01001001
\item Wartość dziesiętna: 73
\end{itemize}
\end{column}
\end{columns}

\note{
Dla każdego piksela porównujemy jego wartość z wartością centralną. Jeśli jest mniejsza - zapisujemy 0, jeśli większa lub równa - zapisujemy 1.
}
\end{frame}

\subsection{Odczyt}

\begin{frame}%{Odczyt spiralny}
\begin{center}
\begin{tikzpicture}[scale=1.2]
% Siatka 3x3
\draw[step=1cm, gray, very thin] (0,0) grid (3,3);

% Numery pozycji
\node at (0.5,2.5) {\Large 1};
\node at (1.5,2.5) {\Large 2};
\node at (2.5,2.5) {\Large 3};
\node at (2.5,1.5) {\Large 4};
\node at (2.5,0.5) {\Large 5};
\node at (1.5,0.5) {\Large 6};
\node at (0.5,0.5) {\Large 7};
\node at (0.5,1.5) {\Large 8};
\node at (1.5,1.5) {\Large \textcolor{red}{X}};

% Strzałki pokazujące kierunek
\draw[->, thick, blue] (0.5,2.8) -- (1.3,2.8);
\draw[->, thick, blue] (1.7,2.8) -- (2.5,2.8);
\draw[->, thick, blue] (2.8,2.5) -- (2.8,1.7);
\draw[->, thick, blue] (2.8,1.3) -- (2.8,0.5);
\draw[->, thick, blue] (2.5,0.2) -- (1.7,0.2);
\draw[->, thick, blue] (1.3,0.2) -- (0.5,0.2);
\draw[->, thick, blue] (0.2,0.5) -- (0.2,1.3);
\end{tikzpicture}
\end{center}

\begin{block}{Zalety odczytu spiralnego}
\begin{itemize}
\item \textbf{Niezmienniczość względem obrotów} - łatwiejsza analiza obrazów obróconych
\item \textbf{Ciągłość przestrzenna} - sąsiednie bity reprezentują sąsiednie piksele
\item \textbf{Uniwersalność} - działa dla dowolnego rozmiaru okna (3×3, 5×5, 7×7, ...)
\end{itemize}
\end{block}

\begin{alertblock}{Uwaga}
Odczyt wierszami jest prostszy w implementacji, ale odczyt „ślimakiem" daje lepsze właściwości dla analizy obrotów obrazu.
\end{alertblock}

\note{
Odczyt spiralny to kluczowa różnica w stosunku do prostego odczytu wierszami. Pozwala na lepszą analizę obrazów obróconych.
}
\end{frame}

\begin{frame}[fragile]{Przykład krok po kroku}
\begin{columns}
\begin{column}{0.5\textwidth}
\textbf{Obraz wejściowy:}
\begin{lstlisting}[language=Python, basicstyle=\tiny\ttfamily]
45  60  30
55  50  40
70  35  65
\end{lstlisting}

\vspace{3mm}

\textbf{Centrum:} 50

\vspace{3mm}

\textbf{Kolejność odczytu:}
\begin{enumerate}
\footnotesize
\item 45 $<$ 50 $\rightarrow$ 0
\item 60 $\geq$ 50 $\rightarrow$ 1
\item 30 $<$ 50 $\rightarrow$ 0
\item 40 $<$ 50 $\rightarrow$ 0
\item 65 $\geq$ 50 $\rightarrow$ 1
\item 70 $\geq$ 50 $\rightarrow$ 1
\item 55 $\geq$ 50 $\rightarrow$ 1
\item 35 $<$ 50 $\rightarrow$ 0
\end{enumerate}
\end{column}
\begin{column}{0.5\textwidth}
\textbf{Wynik:}

\vspace{3mm}

Ciąg bitów: \texttt{01001110}

\vspace{3mm}

Wartość dziesiętna: \textbf{78}

\vspace{5mm}

\begin{alertblock}{Interpretacja}
Liczba 78 to \textbf{deskryptor} tego fragmentu obrazu. Możemy go porównywać z innymi deskryptorami używając odległości Hamminga.
\end{alertblock}
\end{column}
\end{columns}

\note{
To konkretny przykład pokazujący jak krok po kroku obliczamy transformatę Census dla jednego piksela.
}
\end{frame}

\section{Odległość Hamminga}

\subsection{Definicja i zastosowanie}

\begin{frame}{Odległość Hamminga}
\begin{block}{Definicja}
\textbf{Odległość Hamminga} to liczba pozycji, na których dwa ciągi binarne o tej samej długości się różnią.
\end{block}

\pause

\begin{exampleblock}{Przykład}
\begin{center}
\begin{tabular}{lcl}
Ciąg 1: & \texttt{01001001} & \\
Ciąg 2: & \texttt{01\textcolor{red}{1}0100\textcolor{red}{0}} & \\
\hline
Różnice: & \multicolumn{2}{c}{\textcolor{red}{2 pozycje (bit 3 i bit 7)}} \\
\end{tabular}
\end{center}

Odległość Hamminga = \textbf{2}
\end{exampleblock}

\pause

\begin{block}{Zastosowanie w Census}
\begin{itemize}
\item \textbf{Mała odległość} ($\approx$ 0-5) $\rightarrow$ regiony są \textbf{podobne}
\item \textbf{Średnia odległość} ($\approx$ 6-15) $\rightarrow$ regiony \textbf{częściowo podobne}
\item \textbf{Duża odległość} ($>$ 15) $\rightarrow$ regiony są \textbf{różne}
\end{itemize}
\end{block}

\note{
Odległość Hamminga to prosta, ale potężna metryka do porównywania ciągów bitowych. Im mniejsza odległość, tym bardziej podobne są regiony obrazu.
}
\end{frame}



\section{Zadania do wykonania}

\subsection{Przegląd zadań}

\begin{frame}{Zadania laboratoryjne}
\begin{block}{Zadanie 1: Transformata Census dla pojedynczego piksela}
Zaimplementuj funkcję \texttt{census\_pixel(image, x, y, window\_size)} obliczającą wartość Census dla jednego piksela.
\end{block}

\pause

\begin{block}{Zadanie 2: Transformata Census dla całego obrazu}
Zaimplementuj funkcję \texttt{census\_transform(image, window\_size)} obliczającą Census dla wszystkich pikseli obrazu (z pominięciem brzegów).
\end{block}

\pause

\begin{block}{Zadanie 3: Odległość Hamminga}
Zaimplementuj funkcję \texttt{hamming\_distance(a, b)} obliczającą liczbę różniących się bitów.
\end{block}

\pause

\begin{block}{Zadanie 4: Mapa podobieństwa}
Zaimplementuj funkcję \texttt{similarity\_map(image1, image2, window\_size)} tworzącą mapę różnic między dwoma obrazami.
\end{block}

\pause

\begin{block}{Zadanie 5: Wyszukiwanie wzorca}
Zaimplementuj funkcję \texttt{find\_pattern(image, template, window\_size, threshold)} wyszukującą wzorzec na obrazie.
\end{block}

\note{
Pięć zadań o rosnącym stopniu trudności. Każde kolejne wykorzystuje poprzednie funkcje.
}
\end{frame}

\subsection{Wskazówki implementacyjne}



\begin{frame}[fragile]{Zadanie 1: Wskazówki}
\begin{block}{Struktura funkcji \texttt{census\_pixel}}
%\onslide<1>
\begin{lstlisting}[language=Python, basicstyle=\tiny\ttfamily]
def census_pixel(image, x, y, window_size=3):
    # TODO: Zaimplementuj obliczanie transformaty Census dla pojedynczego piksela
    # 1. Oblicz promień okna: radius = window_size // 2
    # 2. Pobierz wartość piksela centralnego
    # 3. Zainicjuj census_value = 0
    # 4. Dla każdego sąsiada w oknie:
    #    - Przesuń census_value o 1 bit w lewo
    #    - Jeśli sąsiad >= centrum, dodaj 1 (census_value |= 1)
    # 5. Zwróć census_value
    offset = window_size // 2
    center_value = image[y,x]
    census_value = 0
    
    for dy in range(-offset, offset +1):
        for dx in range(-offset, offset +1):
            if dy == 0 and dx == 0:
                continue

            census_value <<= 1
            
            if image[y+dy, x+dx] >= center_value:
                census_value |= 1
    return census_value
\end{lstlisting}
\end{block}

%\onslide<2>
\begin{alertblock}{Kluczowe operacje bitowe}
\begin{itemize}
\item \texttt{<<=} - przesunięcie bitowe w lewo (zrobienie miejsca na nowy bit)
\item \texttt{|=} - bitowe OR (ustawienie bitu na 1)
\end{itemize}
\end{alertblock}


\end{frame}

\begin{frame}[fragile]{Zadanie 2: Wskazówki}
\begin{block}{Struktura funkcji \texttt{census\_transform}}
\begin{lstlisting}[language=Python]%, basicstyle=\tiny\ttfamily]
def census_transform(image, window_size=3):
    # TODO: Zaimplementuj transformatę Census dla całego obrazu
    # 1. Utwórz macierz wynikową wypełnioną zerami (dtype=np.uint64)
    # 2. Oblicz promień okna
    # 3. Dla każdego piksela (z pominięciem brzegów):
    #    - Wywołaj census_pixel i zapisz wynik
    # 4. Zwróć macierz wynikową
    rows,cols = image.shape
    census = np.zeros((rows, cols), dtype=np.uint64)
    offset = window_size // 2

    for y in range(offset, rows - offset):
        for x in range(offset, cols - offset):
            census[y, x] = census_pixel(image, x, y, window_size)

    return census
\end{lstlisting}
\end{block}
\end{frame}


\begin{frame}[fragile]{Zadanie 3: Wskazówki}
\begin{lstlisting}[language=Python, caption={Prosta implementacja w Pythonie}]
def hamming_distance(a, b):
    """
    Oblicza odległość Hamminga między dwoma liczbami.
    
    Parametry:
    a, b - liczby całkowite (reprezentacje Census)
    
    Zwraca:
    int - liczba różniących się bitów
    """
    xor_result = a ^ b  # XOR - różnice bitów
    return bin(xor_result).count('1')  # Zlicz jedynki

# Przykład użycia
census1 = 0b01001001  # 73
census2 = 0b01101000  # 104
distance = hamming_distance(census1, census2)
print(f"Odległość Hamminga: {distance}")  # Wynik: 2
\end{lstlisting}

\begin{alertblock}{Kluczowa operacja}
Operator \texttt{XOR} (\texttt{\^}) zwraca 1 tam, gdzie bity się różnią, i 0 tam, gdzie są takie same.
\end{alertblock}
\end{frame}


\begin{frame}[fragile]{Zadanie 4: Mapa podobieństwa}
\begin{block}{Idea}
Dla każdego piksela oblicz odległość Hamminga między wartościami Census z dwóch obrazów. Im większa odległość, tym większa różnica.
\end{block}

\begin{lstlisting}[language=Python, basicstyle=\tiny\ttfamily, caption={Szkielet funkcji}]
def similarity_map(image1, image2, window_size=3):
    # 1. Oblicz Census dla obu obrazów
    census1 = census_transform(image1, window_size)
    census2 = census_transform(image2, window_size)
    
    # 2. Utwórz mapę różnic
    similarity = np.zeros_like(census1, dtype=np.uint8)
    
    # 3. Dla każdego piksela oblicz odległość Hamminga
    for y in range(census1.shape[0]):
        for x in range(census1.shape[1]):
            similarity[y, x] = hamming_distance(
                census1[y, x], census2[y, x]
            )
    
    return similarity
\end{lstlisting}

\begin{exampleblock}{Wizualizacja}
Jasne piksele = duże różnice, ciemne piksele = małe różnice
\end{exampleblock}

\note{
Mapa podobieństwa to wizualizacja różnic między obrazami. Przydatna do detekcji zmian.
}
\end{frame}

\begin{frame}[fragile]{Zadanie 5: Wyszukiwanie wzorca}
\begin{block}{Algorytm sliding window}
\begin{enumerate}
\item Oblicz Census dla obrazu i wzorca
\item Przesuń wzorzec po całym obrazie
\item Dla każdej pozycji oblicz średnią odległość Hamminga
\item Zwróć pozycje, gdzie odległość $<$ threshold
\end{enumerate}
\end{block}

\begin{lstlisting}[language=Python, basicstyle=\tiny\ttfamily, caption={Fragment kluczowy}]
# Wytnij fragment obrazu o rozmiarze wzorca
region = census_img[y:y+tpl_h, x:x+tpl_w]

# Oblicz średnią odległość Hamminga
total_distance = 0
count = 0
for ty in range(tpl_h):
    for tx in range(tpl_w):
        total_distance += hamming_distance(
            region[ty, tx], census_tpl[ty, tx]
        )
        count += 1

avg_distance = total_distance / count
if avg_distance < threshold:
    matches.append((x, y, avg_distance))
\end{lstlisting}

\note{
To najbardziej zaawansowane zadanie. Wymaga zrozumienia wszystkich poprzednich koncepcji.
}
\end{frame}

\section{Podsumowanie}

\begin{frame}{Co dziś osiągniesz?}
\begin{block}{Wiedza teoretyczna}
\begin{itemize}
\item Rozumiesz ideę transformaty Census
\item Znasz różnicę między odczytem wierszowym a spiralnym
\item Wiesz czym jest odległość Hamminga i jak ją stosować
\item Rozumiesz zastosowania Census w przetwarzaniu obrazów
\end{itemize}
\end{block}

\begin{block}{Umiejętności praktyczne}
\begin{itemize}
\item Potrafisz zaimplementować transformatę Census
\item Umiesz obliczać odległość Hamminga
\item Potrafisz tworzyć mapy podobieństwa między obrazami
\item Umiesz wyszukiwać wzorce na obrazach
\end{itemize}
\end{block}

\begin{alertblock}{Kluczowa zaleta Census}
Transformata Census jest \textbf{odporna na zmiany oświetlenia}, co czyni ją idealną do porównywania obrazów w różnych warunkach.
\end{alertblock}

\note{
Transformata Census to potężne narzędzie. Dziś nauczyłeś się zarówno teorii, jak i praktycznej implementacji.
}
\end{frame}

\begin{frame}{Pytania i odpowiedzi}
\begin{center}
\Huge Pytania?

\vspace{1cm}

\Large Powodzenia w implementacji!

\vspace{1cm}

\normalsize
Materiały dostępne na:
\texttt{https://github.com/kbor89/TWiPO}
\end{center}

\note{
Czas na pytania studentów. Upewnij się, że wszyscy rozumieją podstawowe koncepcje przed rozpoczęciem pracy.
}
\end{frame}

\end{document}